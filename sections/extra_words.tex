\textit{NOTE TO DO: Epidemiology models don’t show their model prediction, but that the factors are statistically significance. Check how to present this.}


In Methods SHAP section: 
We also illustrated our model with \textit{prototype} patients which vary patient characteristics in a controlled way.
 
\section{Data}


Input feature representation:
\begin{enumerate}
    \item The patients covid status records whether a they had a covid infection on the date of event (defined as within 30 days), and/or are in the covid vaccinated cohort (defined as within 1 year).
    \item We defined acute stroke treatment as a categorical feature with 8 levels, using 4.5 hours from onset of stroke to treatment as the early/late cutoff: no treatment, only early IVT, only late IVT, early IVT and MT, early IVT and late MT, late IVT and late MT, only early MT, only late MT.
    \item Combined onset in hospital and onset to arrival within 6 hours into a single categorical feature "onset in hospital or arrival within 6 hours".
    \item Categorical features are represented as one hot encoded features (ethnicity, covid status, treatment time, onset in hospital or arrival within 6 hours, stroke type, onset time type).
\end{enumerate}


The three populations will be referred to as: i) the \emph{full exploratory population} (388,347 patients), ii) the \emph{causal population} (136,397 patients), and iii) the \emph{matched population} (236,632 patients).

The full dataset contained 52 features that described patient characteristics, acute stroke pathway timings, and use/time to thrombolysis and/or thrombectomy. 


The framework to transfer SSNAP data to the NHSE SDE focused on capturing the first 72 hours of acute stroke care, therefore it is not reliable to use features in the SSNAP dataset held in the NHSE SDE that record data beyond this timeframe, nor data recorded about hospital transfers or hospital attended. Accounting for these limitations, 52 features were provided for the acute phase of the stroke pathway, up to and including any acute treatment (with thrombolysis and/or thrombectomy). 


# EXTRA WORDS

Using the framework of causality, we define our causal model to isolate the affect of acute stroke treatment on the location spent in the first year following stroke, as needing to account for 5 features (those that affect both the treatment choice and the outcome): 1. prior disability; 2. stroke severity on arrival; 3. age; 4. any atrial fibrilation diagnosis, 5. precise onset known. 


\subsection{How do key features impact the location spent in the year following a stroke?}

Notebooks 032jkl

We fit three XGBoost models (on the full population) to predict each of the location options (time in care, time at home, time dead) in the year following the stroke from these 10 input features (features are included if they were used in S2, informed by the feature selection, or those that we are interested in: 1. prior disability; 2. stroke severity on arrival; 3. age; 4. any atrial fibrilation diagnosis, 5. precise onset known 6. treatment; 7. ethnicity, 8. index of multiple deprivation, 9. covid vaccination or infection, 10. diabetes). The SHAP values for each feature describe the relationship with the time spent in each location.


\subsection{The causal affect of acute stroke treatment on the location spent in the first year following a stroke}.

Notebooks 032mno

Using the framework of causality, we define our causal model to isolate the affect of acute treatment on the location spent in the first year following stroke, as needing to account for 5 features (those that affect both the treatment choice and the outcome): 1. prior disability; 2. stroke severity on arrival; 3. age; 4. any atrial fibrilation diagnosis, 5. precise onset known). We limit the population as those patients that can be considered for treatment, so those that have their onset out of hospital and arrive in time for treatment (within 6 hours of onset), with an ischaemic stroke.

In order to isolate the intervention of giving acute stroke treatment. Acute stroke treatment is defined as a categorical feature with 8 levels (using 6 hours as the threshold for early/late treatment): no treatment; early IVT only; late IVT only; early MT only; late MT only; early IVT and early MT; early IVT and late MT; late IVT and late MT. We removed the n patients that had late IVT and early MT as this was deemed to be more likely a data entry error than a valid treatment option.

We fit three XGBoost models to predict each of the location options (time in care, time at home, time dead) in the year following the stroke from the 8 input features. The SHAP values for the treatment feature describe the causal relationship of treatment with time spent in each location.

\textbf{ARE WE DOING THIS NEXT ANALYSIS (counterfactuals of whether or not give treatment)? Look at this afterwards, as we may need to divide the population, and use all the population, to those that are given IVT, and not and those that are given MT and not and report the mean SHAP value for these populations.}

\subsection{Counterfactual treatment: Predicting if a patient would be likely to spend less time in care with or without thrombolysis and/or thrombectomy}

We used the first k-fold \textit{thrombolysis outcome} model to calculate the counterfactual outcome for each treatment option (with and without thrombolysis) for the ???? patients in the first k-fold test set. We represented patients as receiving thrombolysis by setting the feature \textit{onset to thrombolysis time} to the sum of the recorded pathway durations for the patient (\textit{onset to arrival} and \textit{arrival to scan}) and added on the median \textit{scan to thrombolysis time} of their attended hospital. The median \textit{onset to thrombolysis time} for the ??? patients is ??? minutes (range ?? to ?? minutes). We represented patients as not receiving thrombolysis by setting the feature \textit{onset to treatment time} to 9,999 minutes. We translated the resulting four predicted number of days spent in care (no treatment, IVT only, MT only, IVT and MT) into two binary features to represent whether the patient was predicted to use less healthcare resources receiving thrombolysis and thrombectomy.


We used stratified 5-fold cross-validation to test the accuracy of the model, for feature selection, and to test reproducibility of patterns detected. The results from the model fitted on the first k-fold split were used to investigate the relationships between feature values and their contribution to the prediction.

Repeat this analysis using the full exploratory population, and the causal population.



Stroke patients were matched to a control patient from the primary healthcare data that lives in an English LSOA, is alive on the date that their matched stroke patient had their stroke event (referred to onwards as the "date of event"), does not have any history of stroke prior to the date of event recorded in the primary healthcare and HES dataset, and is matched to the stroke patient on age (5 year age band), sex (male or female) and ethnicity (categorised using the 5 categories: white; black, black british, african and carribean; mixed or multiple ethnic groups; asian or british asian; other ethnic group). 


In 5-fold cross validation the data is split into 5 different 80:20 train/test splits with each patient being in one, and only one, test set. 



LIMITATIONS
This was one of the first projects to use the SSNAP dataset as held in the NHSE SDE. 


Have 388,506 patients in SSNAP that had a stroke and attended an acute stroke unit in England between November 2019 and March 2023.

For the analysis requiring a control (descriptive stats and survival analysis and Cox and Kaplan) describe the matching process (Tom and Jadenes work). Remove from the dataset any patient that does not have a control match (after the ?? iterations of trying to) (n=??), the patients that have a gender label mismatch (n=??), and patients that have a discrepency in their recorded age (SSNAP age recorded by year, CONTROL recorded by 5 year ranges, allow a buffer of +/- 4 years difference in the recorded age) (n=??). After these filters keep 98.16\% patients (381,340 patients).


138,418 patients live in an English LSOA, have an ischaemic stroke onset out of the hospital and arrive at hospital within 6 hours (and do not have MT before IVT). [from notebook 039a]

??? patients live in an English LSOA, have a stroke and arrive at one of the n English acute stroke hospitals (and do not have MT before IVT). [from notebook ???]

All analysis was performed in the NHS England Secure Data Environment, using seven of the available datasets: Sentinel Stroke National Audit Programme (SSNAP), demographics, Lower Super Output Area (LSOA), primary healthcare, Hospital Episode Statistics (HES), hometime, and covid.
We were one of the first projects to use the SSNAP dataset as held in the NHSE SDE. The data transfer framework was set up with the aim to capture the first 72 hours of acute stroke care. Due to inconsistencies beyond this timeframe, it is not reliable to use features in the SSNAP dataset held in the NHSE SDE that record data beyond this timeframe. It was also not possible to accurately use data that was recorded about hospital transfers, or hospital attended.

Data were retrieved for ??? emergency stroke admissions coming from an English LSOA, that that arrived at an acute stroke team in England after their stroke onset, obtained from SSNAP from November 2019 til March 2023. SSNAP prospectively collects clinical data from 100\% of acute hospitals in England and Wales, with case ascertainment estimated at $>$90\% when compared with administrative coding data. The data includes ??? English acute stroke teams. Data fields were provided for the acute phase of the stroke pathway, up to and including any acute treatment (with thrombolysis and/or thrombectomy). The index of multiple deprivation quintile for each LSOA is obtained from the Office for National Statistics. The number of days in the year following the date of stroke event spent in a healthcare setting is obtained from HES, date patient died is obtained from ??, and the duration spent at home from hometime. Demographic data (ethnicity, age, sex, LSOA) is obtained from the Office for National Statistics. Primary healthcare data (containing patients age, sex and ethnicity) is used to create the control patient cohort.

We use 30 days to define whether a patient had a covid infection at the time of their stroke event, and 12 months to define whether a patient is in the covid vaccinated cohort.

(notebook: CCU085_01-D11b-covid_vacc)



Analysis in this paper uses 3 different populations of patients. They will be referred to as: i) the exploratory population (388,347 patients), ii) the matched population (236,632 patients), and iii) the causal population (136,397 patients). 

The three populations are created by applying filters to the 388,506 incidences of a stroke arriving at an English acute stroke hospital between 01/01/2018 and 31/03/2023 as recorded in SSNAP.

For the exploratory population patients i) have their stroke onset within 7 days of the arrival date range (excluded 133 patients), have MT before IVT (excluded 25 patients), have scan after IVT (excluded 1 patient), have scan after MT (excluded no patients).

For the matched population, in addition to the exploratory population filters, patients also i) have a matched control patient from primary records (excluded 14 patients), ii) stroke patients have age recorded in SSNAP within 4 years of the 5 year age band recorded for control (excluded 3731 patients), iii) stroke patients have same sex recorded in SSNAP and demographics (excluded 3419 patients), iv) same ethnicity recorded between stroke patients and their matched control patients (excluded 0 patients), v) same sex recorded between stroke patients and their matched control patients (excluded 0 patients), vi) arrive from 01/11/2019 to remove survival bias for controls (excluded 144551 patients)

For the causal population, in addition to the exploratory population filters, patients also i) arrive within 6 hours of onset (excluded 208,381 patients), ii) have their stroke onset out of hospital (excluded 19,682 patients), iii) had an infarction (excluded 48,058 patients), iv) onset known (excluded 120,434 patients)

Represent features:
Make stroke type One Hot Encoded (keep unknown in).
Make onset type One Hot Encoded (3 categories: BE, P, NK)
Durations for unknown onset: 9999
Onset unknown then arrival is later than 6 hrs (onset in hospital, arrival early, arrival late / unknown onset)
Time to treatment: early or late / unknown onset (reflects clinical grouping – show there’s recoverable portions of the brain)

\subsection{Control population}
For each of the ??? stroke patients, a control patient (from the primary healthcare data that lives in an English LSOA and is alive on the date that their matched stroke patient had their stroke event, referred to onwards as the "date of event") is matched based on age (5 year age band), sex (male or female) and ethnicity (categorised using the 5 categories: white; black, black british, african and carribean; mixed or multiple ethnic groups; asian or british asian; other ethnic group). A control patient has no history of stroke (as recorded in HES). We used demographic data to match the patients age, however the ages of the stroke patients in SSNAP were not always consistent. We applied a further check and removed any patient pairs that had age recorded in SSNAP that were 4 years outside the 5 year range as recorded in the demographic data.


\section{From methods}
To describe the numober of patients in analysis.

\begin{itemize}


    \item Notebook 041 reads in 01\_out\_ssnap, and the other files containing the hometime & LSOA data for both the stroke and control population. Of the 388,506 stroke patients, 388,492 have a control match (14 stroke patients did not get a match). The data in the demographics dataset was used for matching stroke patient to control. Stroke patients also have age and sex features in the SSNAP dataset. Only keep patients if the stroke patient has a match on age (within 4 yr tolerance) (filtered to 384,760 patients) and filtered to 381,340 same sex. (98.16\% kept) (Note: ethnicity is only in one dataset, demographics, so nothing to compare it to - not included in SSNAP).
    
    On top of this then restrict to patients that have a start date after 01/11/2019 to remove survivor bias in the control patients, as primary care dataset is created from patients that are alive on this date, so prior to this date the controls have a survivor bias [applying 01/11/2019 removes 37.6\% of records, from 381,340 to 236,681 instances [144,659 further instances removed]. Restrict end date to 31/03/2023 so have stroke event a year before the end of the hometime date (31/03/2024) so have full 365 days of information for the patients [applying 31/03/2023 does not remove any further instances]. 

    **KP TO DO: remove the single patient that has IVT late and MT early. Dubious data recorded.**
    
    This notebook has descriptive stats on the two populations spending time in the three locations.
    
    \item 236,681 patients also used for survival analysis notebooks 050+ (notebook 050 reads and joins the data tables, and applies the age, sex, have a control and date filters in a pyspark dataframe which is saved as table in the database as "ccu085_01_out_survival_analysis".
    
    \item 136,418 patients used for causal inference by filtering patients on those that arrive in time to be considered for treatment (arrive within 6 hours, onset out of hospital, ischaemic).(notebooks 032 jkl)
    \item 386,924 patients used for prediction of time in locations.  Filtered for patients with stroke type recorded (note that this filter is not included for the survival analysis and descriptive stats).(notebooks 032 stu).


    
\end{itemize}


\section{Notebooks used}
\begin{itemize}
    \item Dataset 01\_out\_ssnap has 388,506 incidences of a stroke occurring and the patient arrived at an English acute stroke hospital between 01/01/2018 and 31/03/2023.

    Question: should we remove patients with their onset time a long way off their arrival time?
    There are 173 patients that have their onset recorded more than 2 days (? or another buffer) before their arrival. There's some dubiously early ones: 1899, 1918, 132, 2000, 2001, 2002 * 6, 2008, 2010 *2, 2012...

    The flip side, there;s 62 incidences with stoke onset after arrival. These all have their onset in hospital, so valid entries (max is 1462 days after, median 14 days).
    
    \item Notebook 041 reads in 01\_out\_ssnap, and the other files containing the hometime & LSOA data for both the stroke and control population. Of the 388,506 stroke patients, 388,492 have a control match (14 stroke patients did not get a match). The data in the demographics dataset was used for matching stroke patient to control. Stroke patients also have age and sex features in the SSNAP dataset. Only keep patients if the stroke patient has a match on age (within 4 yr tolerance) (filtered to 384,760 patients) and filtered to 381,340 same sex. (98.16\% kept) (Note: ethnicity is only in one dataset, demographics, so nothing to compare it to - not included in SSNAP).
    
    On top of this then restrict to patients that have a start date after 01/11/2019 to remove survivor bias in the control patients, as primary care dataset is created from patients that are alive on this date, so prior to this date the controls have a survivor bias [applying 01/11/2019 removes 37.6\% of records, from 381,340 to 236,681 instances [144,659 further instances removed]. Restrict end date to 31/03/2023 so have stroke event a year before the end of the hometime date (31/03/2024) so have full 365 days of information for the patients [applying 31/03/2023 does not remove any further instances]. 

    **KP TO DO: remove the single patient that has IVT late and MT early. Dubious data recorded.**
    
    This notebook has descriptive stats on the two populations spending time in the three locations.
    
    \item 236,681 patients also used for survival analysis notebooks 050+ (notebook 050 reads and joins the data tables, and applies the age, sex, have a control and date filters in a pyspark dataframe which is saved as table in the database as "ccu085_01_out_survival_analysis".
    
    \item 136,418 patients used for causal inference by filtering patients on those that arrive in time to be considered for treatment (arrive within 6 hours, onset out of hospital, ischaemic).(notebooks 032 jkl)
    \item 386,924 patients used for prediction of time in locations.  Filtered for patients with stroke type recorded (note that this filter is not included for the survival analysis and descriptive stats).(notebooks 032 stu).
\end{itemize}




\section{extra words}

\subsection{Abstract Version 2}

\textit{Introduction}: In an attempt to move acute stroke care in the right direction, NHS England has set a blanket thrombolysis rate target of 20\%. It has previously been found that hospitals differ by their patient population and so a hospital specific rate is more realistic. In addition, lower thrombolysing hospitals have responded to the request for them to increase their rates with the query "Will this cause more harm?".\

\textit{Patients and methods}:  We modelled the disability at discharge for patients with a stroke. We see that compared to the actual thrombolysis rates per hosptial (n to n\%) they can by n to n\% by making the treatment decision outcome based on whether thrombolysis gives a better outcome. \\

\textit{Results}: This change from actual to best decision changes the average population mRS from n to n. The characteristics of the patients that contribute to identifying them as ones that have better outcome with treatment, that currently aren't getting it, are the non ideal values for each characteristic. Conversely, the characteristics of the patients that contribute to identifying them as ones that do not have a better outcome with treatment, that currently are receiving it, are the ideal values. 

\textit{Discussion and Conclusion}: This highlights that it is a complex issue to identify who to give treatment to, and that a blanket rule on a single characteristic value will not capture all of the information. 


From Motivation
Regardless of the efforts spent, the rates have not changed and, nationally, the 20\% target has never been met. 

It has Even accounting for  found The barrier to increasing the thromboylsis rates are that it comes with it's risks. If it does not kill you (by causing a catastrophic bleed) then you will likely have a less severe disability (due to breaking down the clot and restoring blood flow to the brain sooner).

In addition, during communications with individual stroke teams, it became apparent that clinicians had a varied range of enthusiasm levels to giving thrombolysis. This can be based on the culture of the hospital, and informed by the previous experience of administering the drug.

A study on a single acute stroke hospital in SW England identified that pathway speed was a barrier [2014, Toms work]. Thrombolysis needs to be administered within 4 hours of stroke onset, so that the benefit (breaking down the clot and restoring blood flow to the brain) still outweighs the risks (causing a catastrophic bleed). The suggested change, for increasing the thrombolysis rates at this hospital, was to call the stroke team ahead from the ambulance so that the stroke team could prepare for the patient arrival, thus reducing the duration taken from arrival to treatment. This translated into an increase in thrombolysis rates from n% to n%.

This insight was rolled out across all of the hospitals in the region (SW England) [2016 (me and MA)]. It was anticipated that for each hospital a specific part of the pathway speed would be identified as the main barrier to receive. However, in addition to pathway speed, it was observed that differences in staff behaviour affected the thrombolysis rates. For example, some stroke teams will put extra “detective work” to estimate the stroke onset, when other teams will take it as written if someone says “I don’t know when the stroke started”, thus removing the possibility of considering thrombolysis for that patient. 

We have looked at the acute stroke pathway (on an individual hospital level) to try to identify the barriers at a hospital level. We identified that it was not all down to the speed of the pathway, it was also due to personal behaviours (around obtaining the onset time, and also the clinical decision to treat). Some hospitals were just more enthusiastic at giving thrombolysis, whereas others were more cautious.

To explore this variety in decision making, and in an attempt to identify additional patients that hospitals could consider giving thrombolysis to (with the motivation to help to increase the thrombolysis rates to the national target of 20\%), we looked at the effect of rolling out the "high benchmark thrombolysis decision" across all hospitals. The "high benchmark thrombolysis decision" was predicting what the treatment decision would be at the 25 hospitals with the most willingness to give treatment. The main resistance this work met, from both patients and clinicians from the lower thrombolying hospitals, was that how do we know if the high benchmark hospitals are making the correct decision based on patient outcomes.

Since YEAR SSNAP has been collecting data on the patients disability at discharge, recorded as the modified Rankin Scale (mRS). This feature is complete for n\% of patients.

Using this feature, we look to address the question that kept coming our way: "What is the best treatment for the patient"? We can now model at a hospital level, and instead of rolling out the "high benchmark thrombolysis decision" we can roll out the thrombolysis decision that's best for the patient, and analyse the thrombolysis rates that this will obtain.

NHS England has a target of 20\% thrombolysis rates. Currently have range 3\% to 30\%.

Lots of work and effort has gone into trying to increase the rates to reach the national target, which have not succedded in many years - ?? years after the clinical trial meta analysis the rates have not changed??.

The barrier to increasing the thromboylsis rates are that it comes with it's risks. If it does not kill you (by causing a catastrophic bleed) then you will likely have a less severe disability (due to breaking down the clot and restoring blood flow to the brain sooner).

Clinicians therefore have a different enthusiasm to giving this treatment which can be based on the culture of the hospital, and informed by the previous experience of administering the drug.

We have looked at the acute stroke pathway (on an individual hospital level) to find ways of improving it to increase the thromboylsis rates. Wanted to identify the barriers at a hospital level. We identified that it was not all down to the speed of the pathway, it was also due to personal behaviours (around obtaining the onset time, and also the clinical decision to treat). Some hospitals were just more enthusiastic at giving thromboblyisis, and others were more cautious.

To explore this variety in decision making, and in an attempt to identify additional patients that hospitals could consider giving thrombolysis to (with the motivation to help to increase the thrombolysis rates to the national target of 20\%), we looked at the effect of rolling out the "high benchmark thrombolysis decision" across all hospitals. The "high benchmark thrombolysis decision" was predicting what the treatment decision would be at the 25 hospitals with the most willingness to give treatment. The main resistance this work met, from both patients and clinicians from the lower thrombolysing hospitals, was that how do we know if the high benchmark hospitals are making the correct decision based on patient outcomes.

Since YEAR SSNAP has been collecting data on the patients disability at discharge, recorded as the modified Rankin Scale (mRS). This feature is complete for n\% of patients.

Using this feature, we look to address the question that kept coming our way: "What is the best treatment for the patient"? We can now model at a hospital level, and instead of rolling out the "high benchmark thrombolysis decision" we can roll out the thrombolysis decision that's best for the patient, and analyse the thrombolysis rates that this will obtain.

\subsection{\textit{Threshold outcome} model}

\emph{Method}
\textit{Threshold outcome} model [target feature: below a specific mRS threshold at discharge]: A set of binary threshold models to predict whether the disability level on discharge for a patient is at or below each of the mRS levels. 

DATA
Of these patients, ??? arrived within 4h of known stroke onset, and were used in this modelling study. A 4h onset-to-arrival cut-off was used to allow for 30min for scan and thrombolysis to be within the allowed 4.5 onset-to-thrombolysis time.


 This model has two parts:
\begin{enumerate}
    \item For the cohort of patients that were predicted to have a better outcome with treatment, 0 is they did not receive treatment, and 1 is they did receive treatment.
    \item For the cohort of patients that were predicted to not have a better outcome with treatment, 0 is they did receive treatment, and 1 is they did not receive treatment.
\end{enumerate}



\subsection{Feature selection}
 – those that sequentially lead to the most improvement in ROC AUC score to predict the patients disability level at discharge. 

The full dataset contains 55 features. In order to simplify the model (for enhanced explainability) we selected a subset of these features to be included in the machine learning model. Features were selected by forward-feature selection (using the k-fold model), identifying one feature at a time that led to the greatest improvement in accuracy as measured by the Receiver Operating Characteristic (ROC) Area Under Curve (AUC) (using the one vs rest approach for multi-0class classification models). We repeated this process, identifying the most important feature to add sequentially, until 25 features were selected. These results were used to identify the number of features to include in our machine learning models. We used the same set of features for the "made the correct treatment decision" model.

Using SHAP values, we revealed patient characteristics that contribute to a patient having the wrong treatment decisions.

We predicted which patients should receive thromboylsis based on whether treatment gave them a better outcome, which we defined (cautiously) as: 



\textit{Treatment decision matches predicted best treatment decision} model



Create two subpopulations based on whether the patient has a predicted better outcome with thrombolysis, and fit a separate model to each subpopulation to predict whether they received the matching treatment decision (where 0 represents not matching, and 1 represents matching).


\section{The Appendix containing further model accuracy analysis}

It is a well callibrated and reliable model.

\item Getting a prediction for having a better outcome with treatment.   

Use this model to predict the counterfactual for each patient (whether they did or did not receive IVT) to understand what impact has IVT had, and whether it was the best decision for that patient.

Using the two probability distributions for disability at discharge, can predict whether each patient has a better outcome receiving treatment. For those patients that received thrombolysis, set the feature onset to treatment time to -100 to predict what their disability at discharge probability is without treatment, and for those patients that were not given thrombolysis (hence do not have a scan to thrombolysis time recorded) assume it is the median of the attended hospital.


\textbf{Basing the paper on the story to get to this chart.}
Using this paper to help the flow: https://journals.sagepub.com/doi/full/10.1177/23969873231189040


This model is used to predict the counterfactual outcome for each patient (whether they did, and did not, receive thrombolysis). For those patients that received thrombolysis, create the counterfactual case by setting the feature onset to treatment time to -100 (to predict what their disability at discharge probability is without treatment). For those patients that were not given thrombolysis create the counterfactual case by assuming their scan to thrombolysis time (which is not recorded) is the median of the attended hospital, and set the feature onset to treatment time the sum of the pathway durations (onset to arrival [recorded], arrival to scan [recorded], scan to thrombolysis [median of attended hospital]). This gives two probability distributions for each patient: a probability across the range of mRS scores at discharge for when the patient receives thrombolysis; and another for not receiving thrombolysis. 

Using a cautious definition for what defines a better outcome with treatment: on average reduce disability, without increasing the risk of death and the severest disability (mRS5+), we can compare these two probability distributions for disability at discharge, and translate it into a binary feature for each patient to represent whether they are predicted to have a better outcome receiving thrombolysis.



SHAP methods section (removed words while threshold model not included in paper):
    2) a patient to have a disability beneath a mRS threshold on discharge, 3)....



\section{Words to describe the counterfactural method}
Using the two probability distributions for disability at discharge, can predict whether each patient has a better outcome receiving treatment. For those patients that received thrombolysis, set the feature onset to treatment time to -100 to predict what their disability at discharge probability is without treatment. For those patients that were not given thrombolysis (hence do not have a scan to thrombolysis time recorded) assume it is the median of the attended hospital.

For those patients that were not given thrombolysis create the counterfactual case by assuming their scan to thrombolysis time (which is not recorded) is the median of the attended hospital, and set the feature onset to treatment time the sum of the recorded pathway durations (onset to arrival and arrival to scan) and the median scan to thrombolysis time of the attended hospital.

Use \textit{Multiclass outcome} model to predict the probability distribution of the patients disability level at discharge for the counterfactual case: whether they did, or did not, receive thrombolysis. Use this to predict what impact receiving thrombolysis would have, and whether it was the best decision for that patient.


\section{method}

\subsection{Sensitivity and specificity}

Calculate sensitivity and specificity by converting the proportion of patients to log odds, add on the contribution of the SHAP values and reversing the log and converting back to proportion.  For sensitivity use the proportion of patients treated and with a good outcome that got treatment, and the SHAP values from the model predicting a good outcome with treatment and treated. For specificity use the proportion not treated and not a good outcome with treatment, and the SHAP values from the model predicting a bad outcome with treatment and not treated.

    Odds = P/ (1-P)
    Adjusted\_log\_odds = log(Odds) + SHAP
    Adjusted\_odds = exp(Adjusted\_log\_odds)
    Result = Adjusted\_odds / (1 + Adjusted\_odds)


    where (for sensitivity) P = proportion\_good\_treated, SHAP = shap\_values\_good\_treated, Result = sensitivity
    where (for sensitivity) P = proportion\_bad\_not, SHAP = shap\_values\_bad\_not, Result = specificity



convert the attended hospital mean SHAP values for predicting if a patient who was predicted to benefit from treatment got treatment, to sensitivity. 

calculate the specificity and sensitivity of treatment. To aid interpretability, 

to sensitivity of treatment (proportion of patients who would benefit from treatment who are treated), and specificity of treatment (proportion of patients who would not benefit from treatment who are not treated):


****

The dataset contains 21,642 patients that are on atrial fibrillation anticoagulants (12.856\% of the total). Of these, 5.4\% receive thrombolysis (1165 patients). Giving thrombolysis to a patient receivng anticoagualants presents an additioan risk of a catastrophic bleed. To be abel to consider giving thrombolytics to this cohort, additional checks are required, such as time last took medication, and carrying out blood clotting tests. Our dataset does not capture these details, so our model can not learn the nuances around deciphering which of these patients to consider to give thrombolysis. For that reason, we will take the cautious approach to over-ride the outcome prediction that all patients recorded to be taking anticoagulants that were not recorded as receiving thrombolysis, they can not have a predicted better outcome with treatment.

There are 21,642 patients in the dataset (12.9\%) that receive atrial fibrilation anticoagulants, 5.4\% of them receive thrombolysis (1165 patients). Giving thrombolysis to a patient on anticoagulants presents an additional risk of a bleed. Our dataset does not capture the details about additional tests carried out for theis cohort to make a safe decisions. Therefore, patients are included in the data to train the model. That is, the model learns patterns from what happens in the real world. However when it comes to using the model to predict whether a patient has a better outcome with thrombolysis, if a patient is taking atrial fibrilation, we will only use the model decision if that patient received thrombolysis. Otherwise we will take the cautious approach and apply a blank "not a better outcome with thrombolysis" for patients receiving anticoagulants and didn't get treatment. Otherwise we could overestimate the benefit of thrombolysis.
****

\section{Method} 1000

\subsection{Data}

Data were retrieved for 168,347 emergency stroke admissions to 118 acute stroke teams in England and Wales for six years, 2016–2021 (inclusive), obtained from the Sentinel Stroke National Audit Programme (SSNAP). Data fields were provided for the hyper-acute phase of the stroke pathway, up to and including our target feature: disability on inpatient discharge. Disability is recorded in the SSNAP dataset using the modified Rankin Scale (mRS), where mRS 0 represents perfect health and mRS 6 represents death. The data includes 118 acute stroke hospitals (each has at least 250 stroke admissions and delivers thrombolysis to at least 10 patients in the six years).

\subsection{Machine learning models}
For the different analysis included in this paper, we trained two separate XGBoost models with different target features. Each analysis will refer to the model used:
\begin{enumerate}
    \item \textit{Multiclass outcome} model: A multiclass classification model to predict the mRS score at discharge, the output as a probability distribution across the seven mRS levels. A 5-fold cross validation is used to test the accuracy of the model, for feature selection, and to test reproducibility of SHAP values. The model fitted on the first k-fold is used to investigate the relationship between feature values and their contribution to the prediction.
    
    \item \textit{Matching treatment} model: A model to predict whether the patient received the treatment decision that was predicted to give them the better outcome (where the binary target feature value of 0 represents not matching, and 1 represents matching). We train two separate models using two different patient sub-populations: i. patients predicted to have a better outcome with thrombolysis ii. patients not predicted to have a better outcome with thrombolysis. 

\subsection{Counterfactual: Predict if a patient has a better outcome with thrombolysis}
For the patients in the test set that had treatment, using the {multiclass outcome} model we can calculate their counterfactual outcome of not receiving treatment. Comparing their two mRS probability distributions we can categorise each patient as to whether the model predicts that they should receive treatment, where they should receving threatment if they have a better outomce. Defining a better outcome with thrombolysis as having a better likelihood of a lesser disability (summarised by the weighted mRS score) and a reduced likelihood of a bad outcome (mRS5+).

There are 21,642 patients in the dataset (12.9\%) that receive atrial fibrilation anticoagulants, 5.4\% of them receive thrombolysis (1165 patients). Giving thrombolysis to a patient on anticoagulants presents an additional risk of a bleed. Our dataset does not capture the details about additional tests carried out for theis cohort to make a safe decisions. Therefore, for patients on anticoagulants that did not receive thrombolysis, we will take the cautious approach and over-ride the models prediction by always categorising them as not having a good outcome with treatment. Otherwise we could overestimate the benefit of thrombolysis.


\subsection{Patient characteristics that contribute to a wrong treatment decision}
Use SHAP values from the {matching treatment} model to explore what patient characteristics contribute to a wrong treatment decision (based on the predicted that a patient has a good outcome with thromboylsis). This will inform which patients to change behaviour for: those to stop giving thrombolysis to, and those to start identifying for treatment. To aid interpretability, convert the SHAP values to sensitivity of treatment (proportion of patients who would benefit from treatment who are treated), and specificity of treatment (proportion of patients who would not benefit from treatment who are not treated):

        INCLUDE EQUATION HERE


\end{enumerate}

 to predict the required target features from the other feature values that describe the patients pathway, the patients characteristics and the attended hospital. For a new instance (in our case, a new patient) the model predicts based on what it has seen before

 Multiclass classification models are effectively a set of models, one for each target class, that gives the probability for each class (in our case, for each mRS score), such that the sum of the individual probabilities equals one (this is known as softmax). The model, therefore, reports a probability distribution across the seven mRS levels for each patient. Train using a 5-fold train-test cross validation used to test the accuracy of the model, for feature selection, and to test reproducibility of SHAP values. The model fitted on the first k-fold is used to investigate the relationship between feature values and predictions. If a prediction for a single mRS level is required, the model will predict that the patient has the mRS level with the highest probability.

 If a prediction for a single mRS level is required, the model will predict that the patient has the mRS level with the highest probability.

***

\subsubsection{Counterfactual: Predict if a patient has a better outcome with thrombolysis}
For the patients in the test set that had treatment, using the {multiclass outcome} model we can calculate their counterfactual outcome of not receiving treatment (by setting the feature onset to treatment time to 999999 minutes). We can translate the resulting two predicted probability distributions (with and without thrombolysis) into a binary feature to represent whether the patient is predicted to have a better outcome receiving thrombolysis. We use a cautious definition to determine whether a patient has a better outcome with treatment from the two probability distributions, where both of these criteria must be met: 1) on average reduce disability (lower weighted mRS), 2) without increasing the likelihood of a bad outcome (lower risk of death and the severest disability, mRS5+).


 SHAP was calculated for each of our XGBoost models: the \textit{multiclass outcome} and the \textit{matching treatment} model. For each of the models, we examined the relationship between feature values and their corresponding SHAP values. This revealed the patient characteristics that contribute to 1) a patient to have a specific mRS level on discharge, and 2) a patient having a treatment decision opposite to what gives the predicted best outcome.



 There are 21,642 patients in the dataset (12.9\%) that receive atrial fibrilation anticoagulants, 5.4\% of them receive thrombolysis (1165 patients). Giving thrombolysis to a patient on anticoagulants presents an additional risk of a bleed. Our dataset does not capture the additional infomration gathered for this cohort to make an informed treatment decision. Therefore, for patients on anticoagulants that did not receive thrombolysis, we will take the cautious approach and over-ride any models prediction by always categorising them as not having a good outcome with treatment. Otherwise we could overestimate the benefit of thrombolysis.



 
\emph{Results}
The impact of a feature value on the likelihood of death can be divided into subgroups. \ref{fig:mrs_violin_6} shows the effect of time to thrombolysis on likelihood of death for mild strokes (NIHSS0-5) and moderate and severe strokes (NIHSS 6+).


An alternative way to interpret the contributions of each feature to the prediction of a patients discharge disability is to analyse the SHAP values from the \textit{Threshold outcome} model (a set of six binary XGBoost models to predict whether a patient is at least as well as mRS0, mRS1, mRS2. mRS3. mRS4 and mRS5). A positive SHAP value for this model can always be interpreted as contributing to the patient having a more favourable outcome.

\begin{figure}
\subfloat[]{\includegraphics[width = 5in]{./images/080_shap_violin.png}}\\
\caption{}
\end{figure}

\begin{figure}
\subfloat[]{\includegraphics[width = 5in]{./images/080_shap_histogram.png}}\\
\caption{}
\end{figure}






\begin{figure}[!h]
    \centering    
    \includegraphics[width=0.75\textwidth]
    {./images/043_outcome_mrs6_violin_plots.png}\\
    \caption{}
    \label{fig:mrs6_violin}
\end{figure}

\begin{figure}[!h]
    \centering    
    \includegraphics[width=0.75\textwidth]
{./images/053_predict_mrs6_split_by_ss.png}\\
    \caption{The impact of a feature value on the likelihood of death can be divided into subgroups. Here we see the effect of time to thrombolysis on likelihood of death for mild strokes (NIHSS0-5) and moderate and severe strokes (NIHSS 6+).}
    \label{fig:mrs6_violin_split}
\end{figure}




fig 2 caption: The impact of a feature value on the likelihood of death can be divided into subgroups. Here we see the effect of time to thrombolysis on likelihood of death for mild strokes (NIHSS0-5) and moderate and severe strokes (NIHSS 6+).


\begin{figure}[!h]
    \centering    
    \includegraphics[width=0.75\textwidth]{./images/042_xgb_7_features_5fold_waterfall_for_each_class.jpg}\\
    \caption{Waterfall plots showing the influence of each feature on the predicted likelihood of a single patient having each level of mRS at discharge (top left displays mRS 0, through to bottom left displaying mRS 6). Each waterfall plot shows the SHAP base value (E|f(x)|) and the SHAP values for each of the input feature values, the sum of which equates to the overall likelihood for the patient being that mRS score at discharge (f(x)). The individual seven mRS predictions combine to create the mRS probability distribution at discharge for the patient (bottom right).
}
    \label{fig:results_waterfall_extra_words}
\end{figure}



FIGURE LAYOUT ABOVE AND BELOW RATHER THAN SIDE TO SIDE
\begin{figure}
    \centering
    \begin{subfigure}{1\textwidth}
      \centering
      \includegraphics[trim={0 0 0 1.2cm}, clip, width=1\linewidth]    {./images/053_xgb_7_features_1fold_999999_thrombolysis_shap_violin_all_features_for_mRS0}\\
      \includegraphics[trim={0 0 0 1cm}, clip, width=1\linewidth]    {./images/053_xgb_7_features_1fold_999999_hosp_shap_hist_mrs0}\\
      \caption{Predict the likelihood of no disability on discharge (mRS 6)}
      \label{fig:mrs_violin}
    \end{subfigure}%ults
\end{figure}
\begin{figure}\ContinuedFloat
    \centering
    \begin{subfigure}{1\textwidth}
      \centering
      \includegraphics[trim={0 0 0 1.2cm}, clip, width=1\linewidth]{./images/053_xgb_7_features_1fold_999999_thrombolysis_shap_violin_all_features_for_mRS6}\\
      \includegraphics[trim={0 0 0 1cm}, clip, width=1\linewidth]    {./images/053_xgb_7_features_1fold_999999_hosp_shap_hist_mrs6}\\
      \caption{Predict the likelihood of death on discharge (mRS 6)}
      \label{fig:mrs_violin}
    \end{subfigure}%ults    
  \caption{Plots showing the relationship between SHAP values and feature values. Left: Predicting the likelihood of having no disability at discharge (mRS 0). Right: Predicting the likelihood of being dead at discharge (mRS 6). Top: Violin plots showing the relationship between SHAP values and feature values. The horizontal line shows the median SHAP value. The plots are ordered in ranked feature importance (using the mean absolute SHAP value across all instances). Bottom: Histogram showing the frequency of SHAP values for the hospital attended.}
    \label{fig:scatter}
\end{figure}


\section{Don't think including this as excluding high and low benchmark models}


WAS IN AS SECTION 4.5 (AFTER FIG 3 BAR CHARTS OF PROBABILITY DISTRIBUTIONS. THIS IS DISCUSSING CONNECT 4 FIGURE. CAN ALSO BE DISPLAYED AS A DONUT CHART)
\subsection{Comparing treatment decisions of the best outcome vs i) high and ii) low benchmark decisions}
(Including this section will need to include another model for treatment decisions, and take majority vote for the top and bottom 25 hospitals).

High benchmark hospitals are those 25 units with the highest IVT rate. Take majority vote and use that as the clinical decision in the model for all hospitals. This will result in giving thrombolysis to more people who we predict will get the better outcome from thrombolysis, but at the cost of more patients who we predict will not get the best outcome from thrombolysis. There's a trade-off of benefiting more people at a smaller increase in causing potential harm from IVT. 

The orange dots are those patients who could benefit but not receive thrombolysis, so the high benchmark hospital majority vote is still missing some patients that we predict could have a better outcome with thrombolysis.

We can also do the opposite, taking the 25 least IVT units and use their majority vote. We see there are more orange dots (not given IVT but expected to have a good outcome), and more white dots (not given and no benefit), showing that the low benchmark vote misses more people who could get benefit yet does not give treatment to those who would not benefit. Green dots are reduced (receives IVT and good outcome), and purple dots are reduced (receives IVT and bad outcome), showing that they are not giving IVT as much patients, and so miss benefit as well as missing harm.

To get the most benefit you can increase risk of harm. There appears to be a trade-off.


\section{SHAP for matchign treatment when using full dattaset, and not just the test set for first kfold + MT patients}


\textbf{Violin plots}
Illustrate this with a violin plot. Figure \ref{fig:shap_violin_all_benefit_ivt} shows patients who should benefit from treatment. What is it about the patients that means they didn't receive it, but would have benefited from treatment?

\begin{figure}
\subfloat[]{\includegraphics[width = 5in]{./images/218_shap_violin_all_benefit_with_ivt_Lowest_weighted_mrs_and_least_mrs5_6_Actual_treatment.jpg}}\\
\caption{}
\label{fig:shap_violin_all_benefit_ivt}
\end{figure}

\ref{fig:shap_violin_none_benefit_ivt} shows patients who should not benefit from treatment. What is it about the patients that means they received it, but would have benefited from not receiving treatment?

\begin{figure}
\subfloat[]{\includegraphics[width = 5in]{./images/218_shap_violin_none_benefit_with_ivt_Lowest_weighted_mrs_and_least_mrs5_6_Actual_treatment.jpg}}\\
\caption{}
\label{fig:shap_violin_none_benefit_ivt}
\end{figure}

See the points as relative, as zero means it is not shifting model from base SHAP.

Safest bet of the best outcome with treatment, choose a patient with ideal characteristics (these are younger, no AF, early IVT, SS 10-25).

These ideal characteristic values also identify patients that have been given thrombolysis but are predicted to not have the best outcome with treatment. So we can't say always give it to a single certain characteristic value, as need to take into account all characteristics.

So a drug label contains a list of ideal characteristics - if you follow this then you miss benefit. 

In order to recognise those patients who would, and would not, you need a model to combine all the characteristics. It's combination of characteristics, not a single value. No simple rules to recognise patients who should be treated but aren't currently receiving treatment (and vice versa).

On the flip side, non-ideal characteristic values (such as older, AF, later IVT, SS mild or severe) help to identify patients that have not been given thrombolysis but are predicted to have the best outcome with treatment.

All of this points back to needing a model.

The model takes into account prior disability 


VALUES BEFORE ADJUST FOR ATRIAL FIBRILLATION ANTICOAGULANTS.
Of the 36,605 patients that received thrombolysis, 28,243 patients were predicted to have a better outcome with thromboylsis, 8,362 patients were not predicted to have a better outcome with thromboylsis.
Of the 354,789 patients that did not receive thrombolysis, 27,017 patients were predicted to have a better outcome with thromboylsis, 27,772 patients were not predicted to have a better outcome with thromboylsis.




Patients on atrial fibrilation anticoagulants presents an additional risk of a bleed, and so any patient given thrombolysis and on anticoagulants may have had some additional tests. The dataset does not capture this information, therefore, for patients on anticoagulants that did not receive thrombolysis, We take the cautious approach and remove patients on atrial fibrilation anticoagulants that are not given thrombolysis from this analysis, otherwise we could overestimate the benefit of thrombolysis. This effects ??1165 patients (of the ??20,825 patients) that are on anticoagulants.





\iffalse
The venn diagram (figure \ref{fig:venn_actual_best}) shows the overlap of patients that were actually treated and those that were only treated if it was predicted to give them the best outcome.
Venn diagram showing what happens to 100 average patients, each arrive within 4 hours known onset. Did they receive IVT, and compare this with whether it was the best decision for each of them - do this by using the counterfactual model and calculating what would your outcome be with/without IVT). Comforting there's an overlap of who should and who are treated. But those Yellow and blue patients are hard to identify using some binary labels. Decision is shall I risk being in the yellow or blue circle.


\begin{figure}
{\includegraphics[width = 5in]{./images/impress_venn_diagram_for_paper.png}}\\ %218_venn_diagram_slide.png}}\\
\caption{Venn diagram comparing two options for making the decision to give thrombolysis to 100 average patients (each arrive within 4 hours of known onset): 1) the actual thrombolysis decision 2) the models prediction that the best outcome is with thrombolysis.}
\label{fig:venn_actual_best}
\end{figure}

\fi



%{./images/210_xgb_all_data_multiclass_outcome_999999_scatter_criteria_not_treated}%210_better_outcome_criteria_scatter_not_treated.png}

%{./images/210_xgb_all_data_multiclass_outcome_999999_scatter_criteria_treated}%210_better_outcome_criteria_scatter_treated.png}

\iffalse

\subsection{\textit{Matching treatment} models: Accuracy}

The treatment decision matches the predicted best treatment decision. Accuracy for predicting treatment for patients that "should be treated" was 74.3\%, and 83.1\% for predicting treatment for patients that "should not be treated". ROCAUC was 0.788 and 0.814 respectively.

The Appendix contains further model accuracy analysis.

\fi





\section{Method}

\subsection{Data}

Data were retrieved for 168,347 emergency stroke admissions to acute stroke teams in England for six years, 2016–2021 (inclusive), obtained from the Sentinel Stroke National Audit Programme (SSNAP). Data fields were provided for the hyper-acute phase of the stroke pathway, up to and including our target feature: disability on inpatient discharge. Disability is recorded in the SSNAP dataset using the modified Rankin Scale (mRS), where mRS 0 represents perfect health and mRS 6 represents death. The data includes 118 acute stroke hospitals (each has at least 250 stroke admissions and delivers thrombolysis to at least 10 patients in the six years).

\subsection{Machine learning models}
For the different patient outcome analysis included in this paper, we trained two XGBoost models to predict various target features from the other feature values that describe the patients pathway, the patients characteristics and the attended hospital. For a new instance (in our case, a new patient) the model predicts based on what it has seen before. Each analysis in the paper will refer to the model used:
\begin{enumerate}
    \item \textit{Multiclass outcome} model [target feature: mRS score at discharge]: A multiclass classification model to predict the likelihood of each level of disability at discharge for each patient who had a stroke. Multiclass classification models can be thought of as a set of models (one for each target class) that gives the probability for that class (in our case, for each mRS score), such that the sum of the probabilities equals one (known as softmax). The model, therefore, reports a probability distribution across the seven mRS levels for each patient. Train using a 5-fold train-test cross validation used to test the accuracy of the model, for feature selection, and to test reproducibility of SHAP values. The model fitted on the first k-fold is used to investigate the relationship between feature values and predictions. If a prediction for a single mRS level is required, the model will predict that the patient has the mRS level with the highest probability.

    Using the \textit{multiclass outcome} model to predict the counterfactual outcomes for each patient (whether they did, and did not, receive thrombolysis) we can translate the two probability distributions obtained into a binary feature to represent whether the patient is predicted to have a better outcome receiving thrombolysis. For those patients that received thrombolysis, create the counterfactual case by setting the feature onset to treatment time to -100 (to predict what their disability at discharge probability is without treatment). For those patients that were not given thrombolysis create the counterfactual case by assuming their scan to thrombolysis time (which is not recorded) is the median of the attended hospital, and set the feature onset to treatment time as the sum of the recorded pathway durations (onset to arrival and arrival to scan) and the median scan to thrombolysis time of the attended hospital. We use a cautious definition (containing two criteria which both must be met) to determine whether a patient has a better outcome with treatment: 1) on average reduce disability, 2) without increasing the risk of death and the severest disability (mRS 5+).
    
    \begin{itemize}
        \item \textit{Calculate new target feature: Patient received treatment decision predicted to give better outcome}. We can define a binary target feature where 0 represents not matching (predicted vs actual), and 1 represents a match by comparing the prediction of whether a patient would have a better outcome with treatment, with their actual treatment received.
    \end{itemize}
    
    \item \textit{Matching treatment} model [target feature: patient received treatment decision predicted to give better outcome]: A model to predict the binary target of whether the patient received the treatment decision that was predicted to give them the better outcome (where the binary target feature value of 0 represents not matching, and 1 represents matching). We train two separate models using two different patient sub-populations: i. patients with a predicted better outcome with thrombolysis ii. patients not predicted to get a better outcome with thrombolysis. 
    
    Use this model (and their associated SHAP values) to explore what patient characteristics contribute to a wrong treatment decision (based on the predicted best outcome). This will inform which patients to change behaviour for: those to instead leave alone when otherwise, and those to instead give treatment. To aid interpretability, convert the SHAP values to sensitivity of treatment (proportion of patients who would benefit from treatment who are treated), and specificity of treatment (proportion of patients who would not benefit from treatment who are not treated):

        INCLUDE EQUATION HERE
    
\end{enumerate}



\iffalse

Figure \ref{fig:shap_matching_treatment} shows the SHAP values for the two \textit{matching treatment} models, that explain the contribution of the feature values to an incorrect treatment decision (based on the \textit{multiclass outcome} model prediction).

Figure \ref{fig:shap_matching_treatment_should} shows patients who should benefit from treatment. What is it about the patients that means they didn't receive it, but would have benefited from treatment?
Figure \ref{fig:shap_matching_treatment_should_not} shows patients who should not benefit from treatment. What is it about the patients that means they received it, but would have benefited from not receiving treatment?

Interpret the SHAP values as relative - where zero means that the feature value is not contributing to moving the prediction away from the default, base, SHAP value. We know that an ideal patient to get the best outcome with treatment has these ideal characteristics: younger, no atrial fibrilation diagnosis, early treatment,  a moderately mild stroke severity, precisely known onset time, and no prior disability. Figure \ref{fig:shap_matching_treatment_should_not} shows that these ideal characteristic values also identify the patients that have been given thrombolysis but are predicted to not have the best outcome with treatment. So treatment is not always beneficial for patients that satisfy a single characteristic value. We need to take the combination of characteristics into account. Conversely, \ref{fig:shap_matching_treatment_should} shows that the non-ideal characteristic values (such as older, later treatment, mild or severe stroke severity, prior disability, and non precise onset time, each help to identify patients that have not been given thrombolysis but are predicted to have the best outcome with treatment.

\begin{figure}
    \centering
    \begin{subfigure}{.5\textwidth}
      \centering
      \captionsetup{width=.9\linewidth}
      \includegraphics[trim={0 0 0 5cm}, clip, width=0.95\linewidth]    {./images/210_xgb_all_data_multiclass_outcome_999999_shap_violin_all_features_all_benefit_with_ivt}\\
      \label{fig:mrs_violin}
    \end{subfigure}%ults
    \begin{subfigure}{.5\textwidth}
      \centering
      \captionsetup{width=.9\linewidth}
      \includegraphics[trim={0 0 0 5cm}, clip, width=0.95\linewidth]{./images/210_xgb_all_data_multiclass_outcome_999999_shap_violin_all_features_none_benefit_with_ivt}\\%{./images/053_predict_mrs6_split_by_ss.png}\\
%        \caption{Predict the likelihood of death on discharge (mRS 6)}
        \label{fig:mrs6_violin_split}
    \end{subfigure}
    \hfill
    \begin{subfigure}{.5\textwidth}
      \centering
      \captionsetup{width=.9\linewidth}
      \includegraphics[trim={0 0 0 0.1cm}, clip, width=1\linewidth]    {./images/210_xgb_all_data_multiclass_outcome_999999_hosp_shap_hist_should}\\
      \caption{Predicting correct treatment option for patients that are predicted to benefit with thrombolysis.}
      \label{fig:shap_matching_treatment_should}
    \end{subfigure}%ults
    \begin{subfigure}{.5\textwidth}
      \centering
      \captionsetup{width=.9\linewidth}
      \includegraphics[trim={0 0 0 0.1cm}, clip, width=1\linewidth]
{./images/210_xgb_all_data_multiclass_outcome_999999_hosp_shap_hist_should not}\\
      \caption{Predicting correct treatment option for patients that are predicted to not benefit with thrombolysis}
      \label{fig:shap_matching_treatment_should_not}
    \end{subfigure}
  \caption{Plots showing the relationship between SHAP values and feature values. The smaller the SHAP value more that feature values contributes to the incorrect treatment decision (based on our \textit{multiclass outcome} model. Left: Predicting correct treatment option for patients that are predicted to benefit with thrombolysis. Right: Predicting correct treatment option for patients that are predicted to not benefit with thrombolysis. Top: Violin plots showing the relationship between SHAP values and feature values. The horizontal line shows the median SHAP value. Bottom: Histogram showing the frequency of the mean SHAP value for the hospital attended.}
  \label{fig:shap_matching_treatment}
\end{figure}
\fi

From results section for the sens spec:

In order to compare decisions between hospitals we predicted use of thrombolysis, and outcome with and without thrombolysis, for all of our cohort of 15,680 patients at all hospitals. In order to perform this analysis we built a model to predict whether thrombolysis would be used for each patient at each hospital using XGBoost as previously described \cite{pearn_what_2023}. The XGBoost \textit{thrombolysis decision} model had an accuracy of 78.7\% and ROC-AUC of 0.86 (see supplementary material for more details). For each stroke team we calculated the \textit{Sensitivity} (proportion of patients who were predicted to benefit from thrombolysis who were predicted to receive thrombolysis) and \textit{specificity} (proportion of patients who were predicted not to benefit from thrombolysis who were predicted not to receive thrombolysis) of treatment. 
